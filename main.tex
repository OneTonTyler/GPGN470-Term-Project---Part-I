\documentclass[11pt]{article}
\usepackage[utf8]{inputenc}
\usepackage[margin=1in]{geometry}
\usepackage{gensymb}

\usepackage[backend=biber,style=ieee,sorting=none]{biblatex}
\addbibresource{./sources.bib}

\usepackage{lipsum}
\renewcommand{\abstractname}{\Large{Introduction}}

\begin{document}
% ---------------- %
\begin{center}
    \large
    \textbf{GPGN 470A Term Project - Part I}
    
    \vspace{0.4cm}
    \large
    Science Literature Review of GNSS-R to Measure Soil Moisture Content
    
    \vspace{0.4cm}
    Tyler Singleton
    
    \vspace{0.4cm}
    \today
    
    \vspace{0.9cm}
\end{center}
\begin{abstract}
    \normalsize
    \vspace{1em}
    
% Introduction of my application for GNSS-R
Geodetic Global Navigation Satellite System (GNSS) Reflectometry (GNSS-R) is a powerful method to analyze soil moisture (SM). Traditionally, spaceborn methods to measure SM used radiometers \cite{CYGNSS}. Two satellites -- Soil Moisture and Ocean Salinity (SMOS) launched by the European Space Agency (ESA) and Soil Moisture Active and Passive (SMAP) launched by the National Aeronautics and Space Administration (NASA) -- are the main sources for mapping global SM \cite{GNSS_R_Spaceborn_Challenges, Semiempirical_Modeling}. Radiometers detect naturally emitted microwave radiation from the ground \cite{CYGNSS, GNSS_R_Spaceborn_Challenges}, which makes these instruments highly sensitive to background temperature brightness and artificial radio frequency interference (RFI) \cite{GNSS_R_Spaceborn_Challenges}. Both satellites have a spacial resolution of $\sim 40km$ and temporal resolution of $2-3$ days \cite{CYGNSS}. Another spaceborn method to measure SM utilizes monostatic radar systems to detect backscattering signals of emitted microwave radiation \cite{CYGNSS}. However, these systems require complex data processing, have a reduced temporal resolution in comparison to radiometers, and high instrumental cost \cite{GNSS_R_Spaceborn_Challenges}. As accurate measurements of global SM are necessary in modeling land-surface hydrogeology \cite{Semiempirical_Modeling} and processing complex relationships of terrestrial ecosystems \cite{CYGNSS}, a third method that makes use of bistatic radar emerged. Utilizing forward scattered reflected signals of GNSS satellites, SM can be monitored with high spacial and temporal resolution \cite{GNSS_R_Spaceborn_Challenges}.
\end{abstract}

% ---------------- %
% Describe GNSS-R as compared to SMAP
\section{A Case for Spaceborn GNSS-R}
In December of 2016, the Cyclone Global Navigation Satellite System (CYGNSS) was launched with the intent to monitor ocean wind speeds during tropical storms and hurricanes by utilizing the advancing methods in processing GNSS-R signals \cite{GNSS_R_Spaceborn_Challenges}. Although designed to measure GNSS reflections from over the ocean, CYGNSS provides land coverage measurements over the tropics with a temporal resolution of $\sim 1$ day \cite{CYGNSS}. A recent piece by \citeauthor{CYGNSS} present a case for extracting SM from CYGNSS data through demonstrating spaceborn GNSS reflections are sensitive to SM.

L-band radiation is considered ideal for sensing SM because its longer wavelengths allow for reduced attenuation though vegetation \cite{CYGNSS, GNSS_R_Spaceborn_Challenges}. Using CYGNSS Level 3 data, \citeauthor{CYGNSS} initially map a section of northern Pakistan along the Indus River. This proved three assumptions: \textit{(1)} Wetter surfaces have higher dialectic properties and thus higher reflectively than drier surfaces; \textit{(2)} reflective power $P_{r, eff}$ is sensitive to surface roughness; \textit{(3)} areas of dense vegetation will result in reduced $P_{r, eff}$ from volume scattering within the canopy \cite{CYGNSS}. Additionally, this showed CYGNSS data sensitive to changes small surface features, but \citeauthor{CYGNSS} express the true spatial resolution to be unknown. 

To define the relationship between SM and CYGNSS reflectively, \citeauthor{CYGNSS} used L-band observations of the same area from SMAP from May to August. By correlating the observed $P_{r, eff}$ from CYGNSS measurements and SM content from SMAP, \citeauthor{CYGNSS} had a strong temporal correlation of $(r=0.84)$, but average spacial correlation for August $(r=0.45)$ and May $(r=0.65)$. The spacial correlation was considered to be mostly from $P_{r, eff}$'s sensitivity to topography, surface water, and land coverage \citeauthor{CYGNSS}. The authors were able to define a linear relationship between observed $P_{r, eff}$ to SM which compared ``favorably'' to \textit{in suit} SM observations \cite{CYGNSS}. Thus showing, spaceborn GNSS-R can be used to measure SM.


% ---------------- %
% Issues 
\section{Coherent and Incoherent Scattering}
\citeauthor{CYGNSS} showed an interesting solution to measure SM utilizing the reflected GNSS signals from the CYGNSS constellation. However, their approach showed three major pitfalls in the current literature that need further development \cite{GNSS_R_Spaceborn_Challenges}. First, developing a relationship for SM from GNSS-R is highly dependent on ancillary data such as; secondly, most works assume the observed $P_{r, eff}$ to be comprised of mainly coherent scattering; and thirdly, influence of observations angles are usually ignored \cite{GNSS_R_Spaceborn_Challenges}. For this section, I will focus on current literature surrounding coherent and incoherent scattering.

The current work of \textcite{Elevation_Angle_Impact} and \textcite{Semiempirical_Modeling} have demonstrated a mixed contribution of coherent and incoherent scattering within the observed $P_{r, eff}$. \citeauthor{Semiempirical_Modeling} demonstrated this to primarily be determined by the surface roughness and incidence angle. Utilizing a Kirchhoff approximation (KA) model and only adjusting for SM; for surfaces with a low surface roughness and correlation length, $P_{r, eff}$ is mostly coherent \cite{Semiempirical_Modeling}. This conceptually makes sense, as a smooth surface will have a specular reflection. For rough surfaces with a large correlation length, the $P_{r, eff}$ becomes dominated by incoherent scattering \cite{Semiempirical_Modeling}. When holding SM constant and only adjusting for incidence angle, \citeauthor{Semiempirical_Modeling} found incoherent scatting to dominate for lower angles -- $\textless 30\degree$ for smooth surfaces and $\textless 60\degree$ for rough surfaces. From this, the authors note that it is faulty to then use a linear regression from SMAP to estimate SM \cite{Semiempirical_Modeling}. 

\section{Conclusion}
Increasing our temporal and spacial resolution of global SM is critical for further understanding of our water cycle, hydrosphere, terrestrial ecosystem, and climate \cite{CYGNSS, GNSS_R_Spaceborn_Challenges}. Development in SM has significantly increased; however future research is needed in modeling surface reflectively to account for coherent and incoherent scattering, reduce reliance on ancillary data, accounting for effects of angle geometry. \citeauthor{CYGNSS} demonstrated spaceborn capabilities of using existing GNSS constellations to map global SM. As more GNSS constellations are launched, the resolution will only increase, so GNSS-R has a bright future.  

% ---------------- %
% Sources
\clearpage
\printbibliography

\end{document}
