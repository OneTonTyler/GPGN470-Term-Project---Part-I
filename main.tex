\documentclass{article}
\usepackage[utf8]{inputenc}
\usepackage[margin=1in]{geometry}

\usepackage[backend=biber,style=ieee,sorting=none]{biblatex}
\addbibresource{./sources.bib}

\usepackage{lipsum}
\renewcommand{\abstractname}{\Large{Introduction}}

\begin{document}
% ---------------- %
\begin{center}
    \large
    \textbf{GPGN 470A Term Project - Part I}
    
    \vspace{0.4cm}
    \large
    Science Literature Review of GNSS-R to Measure Soil Moisture Content
    
    \vspace{0.4cm}
    Tyler Singleton
    
    \vspace{0.4cm}
    \today
    
    \vspace{0.9cm}
\end{center}
\begin{abstract}
    \normalsize
    \vspace{1em}
    
% Introduction of my application for GNSS-R
Geodetic Global Navigation Satellite System (GNSS) Reflectometry (GNSS-R) is a powerful method to analyze soil moisture (SM). Traditionally, SM has been measured using radiometers \cite{CGYNSS}. Two satellites -- Soil Moisture and Ocean Salinity (SMOS) launched by the European Space Agency (ESA) and Soil Moisture Active and Passive (SMAP) launched by the National Aeronautics and Space Administration (NASA) -- are the main sources for mapping global SM \cite{GNSS_R_Spaceborn_Challenges, Semiempirical_Modeling}. Radiometers detect naturally emitted microwave radiation from the ground \cite{CGYNSS, GNSS_R_Spaceborn_Challenges}, which makes these instruments highly sensitive to background temperature brightness and artificial radio frequency interference (RFI) \cite{GNSS_R_Spaceborn_Challenges}. Both satellites have a spacial resolution of $40km$ and temporal resolution of $2-3$ days \cite{CGYNSS}. Another method to measure SM utilizes monostatic radar systems to detect backscattering signals of emitted microwave radiation \cite{CGYNSS}. However, these systems require complex data processing, have a reduced temporal resolution in comparison to radiometers, and high instrumental cost \cite{GNSS_R_Spaceborn_Challenges}. As accurate measurements of global SM are necessary in modeling land-surface hydrogeology \cite{Semiempirical_Modeling} and processing complex relationships of terrestrial ecosystems \cite{CGYNSS}, a third method that makes use of bistatic radar emerged. Utilizing forward scattered reflected signals of GNSS satellites, SM can be monitored with high spacial and temporal resolution \cite{GNSS_R_Spaceborn_Challenges}.
\end{abstract}

% ---------------- %
% Describe GNSS-R as compared to SMAP
\section{How GNSS-R is used to Measure Soil Moisture Content}
The theoretical background of gnss-r is follows by utilizing the reflected l-band through a series of signal to noise processing to determind the moisture content. Wet surfaces are more reflective than dry surfaces, so by taking the amplitude of the spectral radiance, the moisture content can be determined. 

The soil is categorized into four parts. Top soil is 0-20cm deep. This is porous and permablie. Middle soil is about 20-60cm. This is less permeabale than topsoil, and contains root systems of small shrubs, plants, and flowers. The final layer is approximately 60cm to 3m. This layer resides just able the bedrock, and is the least permeable layer. It contains tree roots. We will focus only on the first layer as GNSS-R can only penetrate this far. 

It is imporatant to under the theoritical background of GNSS-R. In our ground based measurements, we cannot distingush between incoming electromagnetic radiation and reflected radiation, so we use a ... method.

For spaced based GNSS-R, we detect both. By using the principles of polarization, we can use the phase shift to and amplitude of the waves to meausre soil moisture content. 
% ---------------- %
% Issues 
\section{Case Studies}
Current literature...

% ---------------- %
% Coherent Scattering and Incoherent Scattering
\section{Future of GNSS-R}
More development is needed in...
The future is bright...

% ---------------- %
% Sources
\clearpage
\printbibliography

\end{document}
