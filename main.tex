\documentclass{article}
\usepackage[utf8]{inputenc}
\usepackage[margin=1in]{geometry}

\usepackage[backend=biber,style=ieee,sorting=none]{biblatex}
\addbibresource{./sources.bib}

\usepackage{lipsum}
\renewcommand{\abstractname}{\Large{Introduction}}

\begin{document}
% ---------------- %
\begin{center}
    \large
    \textbf{GPGN 470A Term Project - Part I}
    
    \vspace{0.4cm}
    \large
    Science Literature Review of GNSS-R to Measure Soil Moisture Content
    
    \vspace{0.4cm}
    Tyler Singleton
    
    \vspace{0.4cm}
    \today
    
    \vspace{0.9cm}
\end{center}
\begin{abstract}
    \normalsize
    \vspace{1em}
    
% Introduction of my application for GNSS-R
Geodetic Global Navigation Satellite System (GNSS) Reflectometry (GNSS-R) is a powerful method to analyze soil moisture (SM). Traditionally, SM has been measured using radiometers \cite{CYGNSS}. Two satellites -- Soil Moisture and Ocean Salinity (SMOS) launched by the European Space Agency (ESA) and Soil Moisture Active and Passive (SMAP) launched by the National Aeronautics and Space Administration (NASA) -- are the main sources for mapping global SM \cite{GNSS_R_Spaceborn_Challenges, Semiempirical_Modeling}. Radiometers detect naturally emitted microwave radiation from the ground \cite{CYGNSS, GNSS_R_Spaceborn_Challenges}, which makes these instruments highly sensitive to background temperature brightness and artificial radio frequency interference (RFI) \cite{GNSS_R_Spaceborn_Challenges}. Both satellites have a spacial resolution of $40km$ and temporal resolution of $2-3$ days \cite{CYGNSS}. Another method to measure SM utilizes monostatic radar systems to detect backscattering signals of emitted microwave radiation \cite{CYGNSS}. However, these systems require complex data processing, have a reduced temporal resolution in comparison to radiometers, and high instrumental cost \cite{GNSS_R_Spaceborn_Challenges}. As accurate measurements of global SM are necessary in modeling land-surface hydrogeology \cite{Semiempirical_Modeling} and processing complex relationships of terrestrial ecosystems \cite{CYGNSS}, a third method that makes use of bistatic radar emerged. Utilizing forward scattered reflected signals of GNSS satellites, SM can be monitored with high spacial and temporal resolution \cite{GNSS_R_Spaceborn_Challenges}.
\end{abstract}

% ---------------- %
% Describe GNSS-R as compared to SMAP
\section{A Case for CYGNSS}
In December of 2016, the Cyclone Global Navigation Satellite System (CYGNSS) was launched with the intent to monitor ocean wind speeds during tropical storms and hurricanes by utilizing the advancing methods in processing GNSS-R signals   \cite{GNSS_R_Spaceborn_Challenges}. Although designed to measure GNSS reflections from over the ocean, CYGNSS provides land coverage over the tropics with a temporal resolution of $\sim 1$ day \cite{CYGNSS}. A recent piece by \citeauthor{CYGNSS} present a case for extracting SM from CYGNSS over land.


\citeauthor{CYGNSS} extracted 
% ---------------- %
% Issues 
\section{Coherent and Incoherent Scattering}
Current literature...

% ---------------- %
% Current Papers
\section{Future of GNSS-R}
More development is needed in...
The future is bright...

% ---------------- %
% Sources
\clearpage
\printbibliography

\end{document}
